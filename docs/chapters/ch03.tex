\chapter{用數學寫程式}

\section{輸入數字 $n$ 印出圖形}
\lstinputlisting[language=Python]{src/ch03/01.py}

\section{輸入數字 $n$ 印出底部重疊的雙三角形}
\lstinputlisting[language=Python]{src/ch03/02.py}

\section{輸入數字 $n$ 印出重疊的空心鑽石圖案}
\lstinputlisting[language=Python]{src/ch03/03.py}

\section{輸入數字 $n$ 印出空心鑽石圖案}
\lstinputlisting[language=Python]{src/ch03/04.py}

\section{撰寫程式印出由左到右遞增的金字塔方塊圖案,每個方塊高為 3,寬為 5}
\lstinputlisting[language=Python]{src/ch03/05.py}

\section{撰寫程式印出對稱圖案}
\lstinputlisting[language=Python]{src/ch03/06.py}

\section{撰寫程式輸入數字印出的窗櫺圖案}
\lstinputlisting[language=Python]{src/ch03/07.py}

\section{印出先由大到小,後由小到大的窗櫺圖案}
\lstinputlisting[language=Python]{src/ch03/08.py}

\section{撰寫程式輸入數字印出上下振動的波形圖案}
\lstinputlisting[language=Python]{src/ch03/09.py}

\section{撰寫程式,輸入數字 $n$ 印出圖案}
\lstinputlisting[language=Python]{src/ch03/10.py}

\section{參考上題,輸入數字 $n$ 印出類似燈籠圖案}
\lstinputlisting[language=Python]{src/ch03/11.py}

\section{撰寫程式輸入奇數印出 X 字母方塊圖案}
\lstinputlisting[language=Python]{src/ch03/12.py}

\section{撰寫程式輸入數字印出 M 字母方塊圖案,方塊寬度為 2}
\lstinputlisting[language=Python]{src/ch03/13.py}

\section{同上題設定,但印出「平滑」版本的 M 字母方塊圖案}
\lstinputlisting[language=Python]{src/ch03/14.py}

\section{輸入數字 2 水平點數,利用方程式建構 2 的筆劃印出 2 中有 2 的圖案}
\lstinputlisting[language=Python]{src/ch03/15.py}

\section{撰寫程式輸入數字印出不同放大程度的「中」字圖案}
\lstinputlisting[language=Python]{src/ch03/16.py}

\section{輸入數字 $n$,印出 $n$ 個數字柱子高的雙座山,每個柱子尺寸為 $4\times 3$}
\lstinputlisting[language=Python]{src/ch03/17.py}

\section{輸入數字 $n$,印出 $n$ 個數字柱子高的雙座山,每個柱子尺寸為 $4\times 3$,並使之成對稱圖形}
\lstinputlisting[language=Python]{src/ch03/18.py}

\section{修改行道樹範例使得印出的行道樹由高到矮兩兩排列}
\lstinputlisting[language=Python]{src/ch03/19.py}

\section{修改行道樹範例加印行道樹的倒影}
\lstinputlisting[language=Python]{src/ch03/20.py}

\section{參考行道樹範例,輸入數字 $n$ 印出高腳房舍}
\lstinputlisting[language=Python]{src/ch03/21.py}

\section{修改上題程式,印出高低交錯的高腳房舍}
\lstinputlisting[language=Python]{src/ch03/22.py}

\section{輸入數字 $n$,印出幾何圖案}
\lstinputlisting[language=Python]{src/ch03/23.py}
