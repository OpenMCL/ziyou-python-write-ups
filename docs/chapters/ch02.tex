\chapter{邏輯、條件式與迴圈}

\section{找出三位數的數字和為 10 且數字都不同的所有三位數,例如:325、910,驗證共有 40 個數}
\lstinputlisting[language=Python]{src/ch02/01.py}

\section{撰寫程式印出由 2 到 99 所有數的質因數乘積}
\lstinputlisting[language=Python]{src/ch02/02.py}

\section{輸入兩整數,印出兩數的直式乘積運算式}
\lstinputlisting[language=Python]{src/ch02/03.py}

\section{題目中運算式的每一個中文代表不同的數字,撰寫程式印出滿足的數學式子}
\lstinputlisting[language=Python]{src/ch02/04.py}

\section{撰寫程式印出直式九九乘法表}
\lstinputlisting[language=Python]{src/ch02/05.py}

\section{輸入列數 $n$,印出 W 圖形}
\lstinputlisting[language=Python]{src/ch02/06.py}

\section{輸入列數 $n$,印出 W 數字遞增圖形}
\lstinputlisting[language=Python]{src/ch02/07.py}

\section{輸入數字 $n$,使用一層迴圈印出圖案}
\lstinputlisting[language=Python]{src/ch02/08.py}

\section{輸入方格內最大數字 $n$,印出方形數字分佈圖案}
\lstinputlisting[language=Python]{src/ch02/09.py}

\section{輸入最大數字 $n$,印出菱形數字分佈圖案}
\lstinputlisting[language=Python]{src/ch02/10.py}

\section{輸入數字 $n$,印出 $n$ 個連在一起的鑽石}
\lstinputlisting[language=Python]{src/ch02/11.py}

\section{輸入小山高 $n$,印出三座山,其中左右山同高,中間山為兩倍高}
\lstinputlisting[language=Python]{src/ch02/12.py}

\section{輸入小山高 $n$,印出三座山,右邊山高為 $3n/2$ (進位整數),中間山高為 $2n$}
\lstinputlisting[language=Python]{src/ch02/13.py}

\section{撰寫程式印出螺旋圖形}
\lstinputlisting[language=Python]{src/ch02/14.py}

\section{印出雙螺旋圖形}
\lstinputlisting[language=Python]{src/ch02/15.py}

\section{輸入高 $n$,印出傾斜排列的 $n$ 個方塊圖案}
\lstinputlisting[language=Python]{src/ch02/16.py}

\section{輸入杯高 $n$,印出呈斜線排列的杯子}
\lstinputlisting[language=Python]{src/ch02/17.py}

\section{輸入三角形高 $n$,印出 $n$ 個上下交錯的三角形}
\lstinputlisting[language=Python]{src/ch02/18.py}

\section{輸入高 $n$,印出圖案}
\lstinputlisting[language=Python]{src/ch02/19.py}

\section{輸入數字 $n$,使用四層迴圈印出圖案}
\lstinputlisting[language=Python]{src/ch02/20.py}

\section{同上題,但印出空心數字圖案}
\lstinputlisting[language=Python]{src/ch02/21.py}

\section{輸入奇數 $n$,使用四層迴圈印出圖案}
\lstinputlisting[language=Python]{src/ch02/22.py}

\section{輸入小 X 的列數 $n$ (奇數),印出大 X 旁邊有兩個連在一起的小 X 圖案}
\lstinputlisting[language=Python]{src/ch02/23.py}

\section{同上輸入,印出兩個大 X 之間有兩組連在一起的小 X 圖案}
\lstinputlisting[language=Python]{src/ch02/24.py}

\section{輸入數字 $n$,印出對應的大象圖案}
\lstinputlisting[language=Python]{src/ch02/25.py}
