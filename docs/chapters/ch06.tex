\chapter{字串}

\section{撰寫程式,輸入十萬以內數字,印出對應中文讀法}
\lstinputlisting[language=Python]{src/ch06/01.py}

\section{同上題,輸入千兆數字,印出對應中文讀法}
\lstinputlisting[language=Python]{src/ch06/02.py}

\section{參考一到七字詩範例,改寫程式,印出兩種斜式排列方式}
\lstinputlisting[language=Python]{src/ch06/03.py}

\section{火焰體詩則是自下往上繞著讀,請設定字串儲存詩句,並撰寫程式,去除標點符號後以火焰體詩印出}
\lstinputlisting[language=Python]{src/ch06/04.py}

\section{參考八山疊翠詩範例,修改程式,印出三座山}
\lstinputlisting[language=Python]{src/ch06/05.py}

\section{同上題,修改程式產生倒影}
\lstinputlisting[language=Python]{src/ch06/06.py}

\section{參考八山疊翠詩範例,修改程式,將山挖洞成圖形}
\lstinputlisting[language=Python]{src/ch06/07.py}

\section{是明嘉靖年間鄔景和的詩作,詩名為「游蘇州半山寺」,詩句排列型式為:頂部的兩層為詩句的前四個字,之後詩句的排列是由右半邊輾轉向下,再由左半邊底層迴轉而上。請撰寫程式,讀入包含標點符號的詩句,印出此種型式的八山疊翠詩}
\lstinputlisting[language=Python]{src/ch06/08.py}

\section{參考連環錦纏枝迴文詩範例,改寫程式輸出由「寒」字順時鐘向內螺旋的排列方式}
\lstinputlisting[language=Python]{src/ch06/09.py}

\section{同上題,改寫程式讓詩的文字排列是由右向左,由上到下排列}
\lstinputlisting[language=Python]{src/ch06/10.py}

\section{給定明末清初萬樹所作的鄉居,共含三首七言律詩,三首詩可排列成連結在一起的方環,稱為「方結連環詩」。撰寫程式,讀入三首跨列詩句字串,印出的方結連環圖案}
\lstinputlisting[language=Python]{src/ch06/11.py}

\section{給定清初名醫葉天士所寫的春夏秋冬四季藥名詩 $1$,詩中隱含二十多味中藥名,讀來饒富趣味,請將四首詩句 (含標點符號) 存為字串串列,撰寫程式印出由右到左排列的四季詩句。在每首詩中,字都是斜向排列}
\lstinputlisting[language=Python]{src/ch06/12.py}

\section{給定四首「首尾相接」描寫四季景色的頂針詩,詩都可以回讀,故稱為轉尾連環式。將每首不重複的 16 字存入 p 串列,撰寫程式,將 p 串列的詩句印成型式}
\lstinputlisting[language=Python]{src/ch06/13.py}

\section{將曹操的短歌行存入跨列字串,撰寫程式,印出傳統由右到左的直排排列方式}
\lstinputlisting[language=Python]{src/ch06/14.py}

\section{使用跨列字串儲存蘇東坡的「水調歌頭」,撰寫程式,印出由右到左的傳統中文排列樣式}
\lstinputlisting[language=Python]{src/ch06/15.py}

\section{給定元稹所寫的寶塔詩,題名為:「一字至七字詩:茶」,請撰寫程式讀入原詩句改以直行方式的輸出}
\lstinputlisting[language=Python]{src/ch06/16.py}

\section{飛雁體詩是一種以菱形方式排列的詩句,詩句的讀法為左右開弓, 呈「人」字形,排列有如雁陣,將以上詩句存入字串,撰寫程式印出對應的飛雁體詩型式}
\lstinputlisting[language=Python]{src/ch06/17.py}

\section{輸入包含整數的字串,撰寫程式修改數字,產生五列數字遞增的字串}
\lstinputlisting[language=Python]{src/ch06/18.py}

\section{輸入包含整數的字串,撰寫程式,印出其中的最大數}
\lstinputlisting[language=Python]{src/ch06/19.py}

\section{基本漢字在萬國碼表中是介於 [19968,40870],撰寫程式,每 20 個印出一列,為部份輸出}
\lstinputlisting[language=Python]{src/ch06/20.py}

\section{英文版的小星星歌詞,撰寫程式設定跨列字串存入小星星歌詞,驗證由前兩列任選一字,假設此字的字數為 $n$,則由此字往下數到第 $n$ 個字,找出新字後,再由新字的字數往下數,如此重複以上步驟,直到不能走下去為止,證明最後一定數到同一個字。}
\lstinputlisting[language=Python]{src/ch06/21.py}

\section{參考唐詩在書法中的排列範例,撰寫程式由左到右列印詩句,各詩句的直排字數介於 5 到 10 之間隨意排列}
\lstinputlisting[language=Python]{src/ch06/22.py}

\section{使用葉天士四季藥名詩,撰寫程式以順時鐘螺旋方式呈現詩句如下}
\lstinputlisting[language=Python]{src/ch06/23.py}
