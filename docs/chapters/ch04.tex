\chapter{串列(一)}

\section{撰寫程式,輸入數字 $n$,儲存由中向外遞增的數字於 $2n-1\times 2n-1$ 方形串列,然後僅列印四個如扇葉的圖案}
\lstinputlisting[language=Python]{src/ch04/01.py}

\section{撰寫程式,輸入 $n$,產生一 $n\times n$ 二維串列,使得元素呈現的排列方式}
\lstinputlisting[language=Python]{src/ch04/02.py}

\section{撰寫程式,輸入 $n$,產生 $2n\times 2n$ 的串列,分四塊存入 [1,9] 的亂數後印出如下}
\lstinputlisting[language=Python]{src/ch04/03.py}

\section{分別產生 a、b、c 三個 $n \times n$ 個串列,內存相同的 [1,9] 亂數,將之合併起來成為新的 $n\times 3n$ 串列後印出。為了便於分辨,新串列在列印時,各區塊以空白隔開}
\lstinputlisting[language=Python]{src/ch04/04.py}

\section{輸入列數 $n$,產生一個二維串列,第一列有一個元素,第二列兩個元素,依此類推,以先縱後橫方式設定遞增數字如下後印出}
\lstinputlisting[language=Python]{src/ch04/05.py}

\section{同上輸入,但產生從上而下的迴旋數字排列方式}
\lstinputlisting[language=Python]{src/ch04/06.py}

\section{有兩個 $n\times n$ 的對稱矩陣,元素由 0 或 1 的亂數組成。由於矩陣為對稱關係,兩者都僅存下三角元素。請計算兩對稱矩陣的乘積,印出相乘過程}
\lstinputlisting[language=Python]{src/ch04/07.py}

\section{輸入數字,撰寫程式印出巴斯卡三角形}
\lstinputlisting[language=Python]{src/ch04/08.py}

\section{同上題,但印出上下對稱的圖案}
\lstinputlisting[language=Python]{src/ch04/09.py}

\section{撰寫程式,隨機生成數字組並根據「幾 A 幾 B」的規則輸出判斷結果}
\lstinputlisting[language=Python]{src/ch04/10.py}

\section{撰寫程式驗證 n 顆骰子的點數和機率}
\lstinputlisting[language=Python]{src/ch04/11.py}

\section{撰寫程式驗證擲出 n 個骰子後,$2 \leq n \leq 7$,僅有兩個骰子的點數是一樣的機率}
\lstinputlisting[language=Python]{src/ch04/12.py}

\section{撰寫程式使用亂數函式產生介於 [-5,5] 間的整數,但不包含零,然後列印成形式的直條圖}
\lstinputlisting[language=Python]{src/ch04/13.py}

\section{撰寫程式,使用亂數設定長度,產生的斜條圖}
\lstinputlisting[language=Python]{src/ch04/14.py}

\section{撰寫程式,產生由長到短上下對稱斜線}
\lstinputlisting[language=Python]{src/ch04/15.py}

\section{撰寫程式,輸入數字,以直條型式分解數字成基本數字的組合}
\lstinputlisting[language=Python]{src/ch04/16.py}

\section{參考方塊螺旋數字範例,利用座標轉換將結果印為鑽石螺旋圖案}
\lstinputlisting[language=Python]{src/ch04/17.py}

\section{撰寫程式,輸入數字,印出三角螺旋數字圖案}
\lstinputlisting[language=Python]{src/ch04/18.py}

\section{撰寫程式,輸入數字 $n$,產生 $n$ 個小綠人點陣圖案,輸出時在每個小綠人身體部位顯示 [1,n] 數字,並打亂編號排列}
\lstinputlisting[language=Python]{src/ch04/19.py}

\section{驗證在蒙提霍爾問題中,當來賓改選門後,得到車子的機率將會由原來不更換門的 $1/3$ 增加到 $2/3$}
\lstinputlisting[language=Python]{src/ch04/20.py}

\section{如上題,若拓展為 $n$ 扇門且僅有一扇門後有車子情況下,驗證當來賓改選門後,得到車子的機率將會由原來不更換門的 $1/n$ 增加到 $(n-1)/(n(n-2))$}
\lstinputlisting[language=Python]{src/ch04/21.py}

\section{撰寫程式,輸入數字,印出縱寬各放大兩倍的點矩陣數字}
\lstinputlisting[language=Python]{src/ch04/22.py}

\section{同上題,輸入數字,但印出傾斜的點矩陣數字}
\lstinputlisting[language=Python]{src/ch04/23.py}

\section{撰寫程式,輸入數字,將此數字的點矩陣上下隨意調整位置後列印出來,點陣背景輸出橫線用以模擬數字釘於木板上的效果}
\lstinputlisting[language=Python]{src/ch04/24.py}

\section{撰寫程式,輸入數字,印出點矩陣數字與其倒影}
\lstinputlisting[language=Python]{src/ch04/25.py}

\section{有一新式彈珠臺在檯面上增加了三個 x y 平臺,彈珠若滾到平臺上會直接由隱藏在平臺下方的洞口掉離彈珠臺,不會繼續往下滾。撰寫程式驗證彈珠滾到 A 到 F 各個位置的機率}
\lstinputlisting[language=Python]{src/ch04/26.py}

\section{參考彈珠臺範例,撰寫程式驗證彈珠滾到 A 到 F 各個位置的機率,同時也要將落在此位置的彈珠球是來自上端入口的機率由左到右一併列印,為輸出的內容}
\lstinputlisting[language=Python]{src/ch04/27.py}

\section{撰寫程式印出門扉,門扉數量 ($\in[4,9]$) 與其方向以亂數設定,輸入 $n$ 控制門扉形狀大小}
\lstinputlisting[language=Python]{src/ch04/28.py}

\section{修改行道樹程式使得行道樹的順序隨意排列}
\lstinputlisting[language=Python]{src/ch04/29.py}

\section{參考房舍習題,修改程式使得房舍順序隨意排列}
\lstinputlisting[language=Python]{src/ch04/30.py}

\section{使用 $5\times 5$ 點矩陣數字儲存「中」與「大」兩個中文字的點陣圖形,輸入中文字縱向與橫向的放大倍數,撰寫程式畫出先遞增放大倍數後遞減放大倍數的「中大寶寶」圖形}
\lstinputlisting[language=Python]{src/ch04/31.py}
