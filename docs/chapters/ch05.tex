\chapter{串列(二)}

\section{撰寫程式,模擬橋牌規則發牌,每人的牌面由大到小排列}
\lstinputlisting[language=Python]{src/ch05/01.py}

\section{四個不同多面柱體骰子分別可顯示 1 到 3 點,1 到 4 點,1 到 5 點,與 1 到 6 點等不同點數,撰寫程式模擬隨意甩這四個骰子,驗算四個骰子的點數呈現連續數字的機率}
\lstinputlisting[language=Python]{src/ch05/02.py}

\section{撰寫程式驗證隨意取五張撲克牌,得到順子的機率為 0.0039246 (請留意,要排除同花順的情況)}
\lstinputlisting[language=Python]{src/ch05/03.py}

\section{請根據男童、女童圖形,自行設計相關的點矩陣串列。在程式中加入相關條件印出四肢與身體等符號,同時使用串列定義男女童資料,以亂數控制兒童數量在 [3,10] 之間與各別的男童、女童。執行時,每按 rtn 鍵後即重新設定 kids 串列,然後隨之印出兒童圖案}
\lstinputlisting[language=Python]{src/ch05/04.py}

\section{利用亂數設定 12 生肖的數量,使用直條圖呈現圖形}
\lstinputlisting[language=Python]{src/ch05/05.py}

\section{利用亂數設定 12 生肖的數量,使用橫條圖呈現圖形}
\lstinputlisting[language=Python]{src/ch05/06.py}

\section{利用亂數設定 12 生肖的數量,利用點散佈圖呈現圖形}
\lstinputlisting[language=Python]{src/ch05/07.py}

\section{利用亂數設定 12 生肖的數量,利用極座標呈現圖形}
\lstinputlisting[language=Python]{src/ch05/08.py}

\section{左圖為國旗圖形尺寸,撰寫程式畫出國旗}
\lstinputlisting[language=Python]{src/ch05/09.py}

\section{利用 pylab.fill\_between 函式在 $x\in [-2\pi,2\pi]$ 間畫出 $\sin(x)$ 函數,圖形依照 $y$ 值區分為四個區域,用不同顏色表示}
\lstinputlisting[language=Python]{src/ch05/10.py}

\section{撰寫程式實作牛頓迭代求根法,求解 $f(x) = \cos(x)$ 的近似根,畫出牛頓迭代法迭代五步的過程,假設 $x_0$ 為 0.43}
\lstinputlisting[language=Python]{src/ch05/11.py}

\section{撰寫程式,分別使用 plot 描線與 fill 塗色畫出蝴蝶圖形}
\lstinputlisting[language=Python]{src/ch05/12.py}

\section{利用上一題的蝴蝶曲線參數方程式,並使用 rgba 的顏色代表方式,輸入 $n$,(a) 印出 $n\times n$ 隻各種不同顏色蝴蝶,(b) 同上,但只印出在對角線上的蝴蝶。假設上下相鄰蝴蝶的間距皆設定為 8}
\lstinputlisting[language=Python]{src/ch05/13.py}

\section{請撰寫程式,使用亂數套件隨意設定 角度 $\theta$、縮放倍率 $r$、與位移量 $(dx, dy)$ 隨意產生 10 到 15 隻蝴蝶}
\lstinputlisting[language=Python]{src/ch05/14.py}

\section{設定「中央」兩字的點陣資料,利用前一題座標轉換公式,撰寫程式使用 pylab 繪圖,讓原本應顯示的點都以蝴蝶表示,請自由變更各蝴蝶的旋轉角與縮放比}
\lstinputlisting[language=Python]{src/ch05/15.py}

\section{設定「中央」兩字的點陣資料,撰寫程式利用 pylab 繪圖,讓原本應顯示的點都以多邊形表示,自由變更各多邊形的邊數、顏色、縮放比、旋轉角}
\lstinputlisting[language=Python]{src/ch05/16.py}

\section{輸入螺線寬度 $n$,使用 pylab 畫出的方形旋轉螺線}
\lstinputlisting[language=Python]{src/ch05/17.py}

\section{同上題,利用旋轉公式與 pylab 繪圖成螺線圖案}
\lstinputlisting[language=Python]{src/ch05/18.py}

\section{撰寫程式,輸入時分,用 pylab 畫出對應的時鐘圖形如下}
\lstinputlisting[language=Python]{src/ch05/19.py}

\section{撰寫程式,輸入刻度尺的起始刻度,畫出 10 公分長的刻度尺如下}
\lstinputlisting[language=Python]{src/ch05/20.py}

\section{撰寫程式,輸入三位數,畫出的度量儀度刻度表}
\lstinputlisting[language=Python]{src/ch05/21.py}

\section{參考第四章「中大」雙重點矩陣圖範例與本章畫出來的數字範例,撰寫程式讀入數字,畫出此數字的雙重點矩陣圖,圖形如下,同時各個數字的顏色請使用亂數設定 $(r,g,b)$ 方式處理}
\lstinputlisting[language=Python]{src/ch05/22.py}
